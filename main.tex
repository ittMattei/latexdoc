\documentclass[12pt]{article}
\usepackage[utf8]{inputenc}
\usepackage{amsmath}
\usepackage{geometry}
\usepackage{lmodern}
\geometry{a4paper, margin=2.5cm}

\title{Introduzione alla Statistica per la Regressione Lineare}
\date{}
\begin{document}

\maketitle

\section*{1. Media}
La \textbf{media} (o media aritmetica) è la somma di tutti i valori divisa per il numero dei valori:

\[
\bar{x} = \frac{1}{n} \sum_{i=1}^{n} x_i
\]

\textit{Esempio:} dati i valori 6, 7, 8, la media è:

\[
\bar{x} = \frac{6+7+8}{3} = 7
\]

\section*{2. Varianza}
La \textbf{varianza} misura quanto i dati sono dispersi intorno alla media.

\[
\sigma^2 = \frac{1}{n} \sum_{i=1}^{n} (x_i - \bar{x})^2
\]

\textbf{Formula semplificata:}

\[
\sigma^2 = \frac{\sum x_i^2 - n \cdot \bar{x}^2}{n}
\]

La \textbf{deviazione standard} è la radice quadrata della varianza:

\[
\sigma = \sqrt{\sigma^2}
\]

\section*{3. Covarianza}
La \textbf{covarianza} misura come due variabili variano insieme:

\[
\text{Cov}(X,Y) = \frac{1}{n} \sum_{i=1}^{n} (x_i - \bar{x})(y_i - \bar{y})
\]

\textbf{Formula semplificata:}

\[
\text{Cov}(X,Y) = \frac{\sum x_i y_i - n \cdot \bar{x} \cdot \bar{y}}{n}
\]

\section*{4. Coefficiente di correlazione (di Pearson)}
Misura quanto due variabili sono linearmente correlate. È un numero tra -1 e 1:

\[
r = \frac{\text{Cov}(X, Y)}{\sigma_X \cdot \sigma_Y}
\]

\begin{itemize}
    \item $r = 1$: correlazione perfetta positiva
    \item $r = -1$: correlazione perfetta negativa
    \item $r = 0$: nessuna correlazione lineare
\end{itemize}

\section*{5. Regressione Lineare}
Serve a trovare la \textbf{retta} che meglio rappresenta la relazione tra due variabili.

\[
y = a + bx
\]

Dove:
\[
b = \frac{\text{Cov}(X,Y)}{\text{Var}(X)} \quad \text{e} \quad a = \bar{y} - b\bar{x}
\]

Questa retta è chiamata \textit{retta di regressione} e permette di fare previsioni. È una delle basi dell'\textbf{apprendimento supervisionato} in Intelligenza Artificiale.

\end{document}

